\section{User registration}

The registration of new users is handled by the
\svnweb{ro/ldir/registration/RegistrationWebService}{ro.ldir.registration.RegistrationWebService}
web resource. Registration is a two step process. In a first step, the new user
submits its details to the system. The system will mark the user's status as
``PENDING''. Then, the system will send the new user an email containing a token
used to validate the user's email address. In the second step, when the user
validates the email, the system moves the user in the default role
``VOLUNTEER''. Later, an administrator can change the role of the recently
validated user. The latter procedure is covered in Section~\ref{sec:call:user}.

All calls described throughout this section are sent without authentication.

\subsection{Submitting new user details}

\apilocation{post}{reg/ws}
\begin{apidata}{content}
  A \texttt{user} entity containing the user to register.
\end{apidata}
\begin{apidata}{returns}
  \begin{datalist}
    \replyitem{200}{if the operation succeeds.}
    \replyitem{409}{if the email is already in use.}
    \replyitem{500}{if another error occurs.}
  \end{datalist}
\end{apidata}

\subsection{Validating the user's email}
\apilocation{get}{reg/ws/\param{userId}/\param{key}}
\begin{apidata}{returns}
  \begin{datalist}
    \replyitem{200}{if the operation succeeds.}
    \replyitem{404}{if user does not exist.}
    \replyitem{409}{if the token is incorrect or the user is already validated.}
  \end{datalist}
\end{apidata}

\subsection{Resetting user passwords}

To reset the password of a user, you must first generate a token using the call:

\apilocation{post}{reg/ws/reset?email=\param{email}}
\begin{apidata}{returns}
  \begin{datalist}
    \replyitem{200}{if the operation succeeds.}
    \replyitem{404}{if user does not exist.}
    \replyitem{500}{if another error occurs.}
  \end{datalist}
\end{apidata}

The previous call sends a token (valid for a limited amount of time) to the user
via email. The token must be used in the following call along which you must
pass the new password:

\apilocation{post}{reg/ws/reset/\param{userId}/\param{token}}
\begin{apidata}{content}
  A string containing the new password of the user.
\end{apidata}
\begin{apidata}{returns}
  \begin{datalist}
    \replyitem{200}{if the operation succeeds.}
    \replyitem{404}{if user does not exist.}
    \replyitem{406}{if the token is invalid.}
    \replyitem{500}{if another error occurs.}
  \end{datalist}
\end{apidata}



