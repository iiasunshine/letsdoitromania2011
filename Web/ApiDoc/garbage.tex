\section{Garbage tag management}

This section describes the calls that affect the state of garbage tags inserted
in the system. These are implemented in the
\svnweb{ro/ldir/ws/GarbageWebService}{ro.ldir.ws.GarbageWebService} web
resource.

All calls to this web service must be authenticated using HTTP headers. The
username in the HTTP headers must match a user's email whose security role is
``ADMIN''.


\subsection{Adding a new garbage tag}

This call inserts a new garbage in the system. The owner of the garbage will be
set to match the authenticated user.

The town, county and charting area of the new garbage tag will be set
automatically by the system using the tag's coordinates and the entities stored
in the database.

\apilocation{post}{ws/garbage}
\begin{apidata}{contents}
  A \texttt{garbage} entity describing the new garbage tag.
\end{apidata}
\begin{apidata}{returns}
  The ID of the recently inserted garbage (if successful) and
  \begin{datalist}
    \replyitem{200}{if the operation succeeds.}
    \replyitem{400}{if the garbage coordinates are wrong, i.e., they do not
    belong to any county.}
    \replyitem{500}{if another error occurs.}
  \end{datalist}
\end{apidata}

\subsection{Anonymously adding a new garbage tag}

This call inserts a new garbage in the system by a user that is not
authenticated.


The town, county and charting area of the new garbage tag will be set
automatically by the system using the tag's coordinates and the entities stored
in the database.

\apilocation{post}{garbage/ws}
\begin{apidata}{contents}
  A \texttt{garbage} entity describing the new garbage tag.
\end{apidata}
\begin{apidata}{returns}
  The ID of the recently inserted garbage (if successful) and
  \begin{datalist}
    \replyitem{200}{if the operation succeeds.}
    \replyitem{400}{if the garbage coordinates are wrong, i.e., they do not
    belong to any county.}
    \replyitem{500}{if another error occurs.}
  \end{datalist}
\end{apidata}



\subsection{Getting garbage tag information}

\apilocation{get}{ws/garbage/\param{garbageId}}
\begin{apidata}{returns}
  \begin{datalist}
    \replyditem{A \texttt{garbage} entity describing the garbage tag.}
  \end{datalist}
\end{apidata}

\subsection{Getting a garbage tag's charted area}

\apilocation{get}{ws/garbage/\param{garbageId}/chartedArea}
\begin{apidata}{returns}
  \begin{datalist}
    \replyditem{A \texttt{chartedArea} entity describing the garbage tag.}
  \end{datalist}
\end{apidata}

\subsection{Getting a garbage tag's county}

\apilocation{get}{ws/garbage/\param{garbageId}/county}
\begin{apidata}{returns}
  \begin{datalist}
    \replyditem{A \texttt{countyArea} entity describing the garbage tag.}
  \end{datalist}
\end{apidata}

\subsection{Getting a garbage tag's town}

\apilocation{get}{ws/garbage/\param{garbageId}/town}
\begin{apidata}{returns}
  \begin{datalist}
    \replyditem{A \texttt{townArea} entity describing the garbage tag.}
  \end{datalist}
\end{apidata}

\subsection{Updating a garbage tag}

\apilocation{post}{ws/garbage/\param{garbageId}}
\begin{apidata}{contents}
  A \texttt{garbage} entity describing the new garbage tag.
\end{apidata}
\begin{apidata}{returns}
  \begin{datalist}
    \replyitem{200}{if the operation succeeds.}
    \replyitem{400}{if the garbage coordinates are wrong, i.e., they do not
    belong to any county.}
    \replyitem{500}{if another error occurs.}
  \end{datalist}
\end{apidata}

\subsection{Setting a garbage to be voted}

\apilocation{post}{ws/garbage/\param{garbageId}/toVote}
\begin{apidata}{contents}
  A boolean indicating whether the garbage can be voted.
\end{apidata}
\begin{apidata}{returns}
  \begin{datalist}
    \replyitem{200}{if the operation succeeds.}
    \replyitem{404}{if the garbage does not exist.}
    \replyitem{500}{if another error occurs.}
  \end{datalist}
\end{apidata}
\begin{apidata}{access}
The logged in user must be an ``ADMIN'', ``ORGANIZER'' or ``ORGANIZER\_MULTI''.
\end{apidata}

\subsection{Setting a garbage to be cleaned}

\apilocation{post}{ws/garbage/\param{garbageId}/toClean}
\begin{apidata}{contents}
  A boolean indicating whether the garbage can be cleaned.
\end{apidata}
\begin{apidata}{returns}
  \begin{datalist}
    \replyitem{200}{if the operation succeeds.}
    \replyitem{404}{if the garbage does not exist.}
    \replyitem{500}{if another error occurs.}
  \end{datalist}
\end{apidata}
\begin{apidata}{access}
The logged in user must be an ``ADMIN'', ``ORGANIZER'' or ``ORGANIZER\_MULTI''.
\end{apidata}

\subsection{Voting for a garbage}

A user can vote for a garbage only once. Only a garbage that has been selected
for voting by an organizer can be voted.

\apilocation{post}{ws/garbage/\param{garbageId}/vote}
\begin{apidata}{returns}
  \begin{datalist}
    \replyitem{200}{if the operation succeeds.}
    \replyitem{400}{if the garbage cannot be voted, either because the user has
    already voted for the garbage, or because no organizer selected the garbage
    for voting.}
    \replyitem{404}{if the garbage does not exist.}
    \replyitem{500}{if another error occurs.}
  \end{datalist}
\end{apidata}

\subsection{Deleting a garbage tag}

\apilocation{delete}{ws/garbage/\param{garbageId}}
\begin{apidata}{returns}
  \begin{datalist}
    \replyitem{200}{if the operation succeeds.}
    \replyitem{500}{if another error occurs.}
  \end{datalist}
\end{apidata}



% \subsection{Deleting a garbage tag}
% 
% This method is not implemented yet.
% 
% \apilocation{delete}{ws/garbage/\param{garbageId}}
% \begin{apidata}{returns}
%   \begin{datalist}
%     \replyitem{200}{If the operation succeeds.}
%   \end{datalist}
% \end{apidata}
% \begin{apidata}{access}
% The logged in user must be an ``ADMIN'' or the garbage tag's owner.
% \end{apidata}


\subsection{Setting the status of a garbage tag}

\apilocation{put}{ws/garbage/\param{garbageId}/status}
\begin{apidata}{contents}
  A string denoting the new status of the garbage. Can be either ``CLEANED'' or
  ``IDENTIFIED''.
\end{apidata}
\begin{apidata}{returns}
  \begin{datalist}
    \replyitem{200}{if the operation succeeds.}
    \replyitem{500}{if another error occurs.}
  \end{datalist}
\end{apidata}
\begin{apidata}{access}
The logged in user must be an ``ADMIN'' or the garbage tag's owner.
\end{apidata}


\subsection{Assigning a new image to a garbage tag}

\apilocation{post}{ws/garbage/\param{garbageId}/image}
\begin{apidata}{contents}
  A ``multipart/form-data'' request containing a field ``file'' representing a
  new image to be associated to this garbage tag.
\end{apidata}
\begin{apidata}{returns}
  \begin{datalist}
    \replyitem{200}{if the operation succeeds.}
    \replyitem{404}{if the garbage tag does not exist.}
    \replyitem{500}{if another error occurs.}
  \end{datalist}
\end{apidata}
\begin{apidata}{access}
The logged in user must be an ``ADMIN'' or the garbage tag's owner.
\end{apidata}


\subsection{Getting an image of the garbage tag}

The following  calls retrieves an image associated to a garbage. All garbage
tags have an array of images, and new images are pushed to the tail of the
array. The array length can be obtained through the \texttt{pictures} field of
the \texttt{garbage} entity.  To make any of the following calls, a valid image
index must be known. Valid indexes are from 0 to \texttt{pictureCount} - 1.

To get an image in its original size, call:

\apilocation{get}{ws/garbage/\param{garbageId}/image/\param{image\_index}}
\begin{apidata}{returns}
  \begin{datalist}
    \replyditem{An ``image/jpeg'' file containing the picture.}
    \replyitem{404}{If the garbage tag does not exist or the index is invalid.}
  \end{datalist}
\end{apidata}

To get a lower resolution file, call:

\apilocation{get}{ws/garbage/\param{garbageId}/image/\param{image\_index}/display}
\begin{apidata}{returns}
  \begin{datalist}
    \replyditem{An ``image/jpeg'' file containing the picture.}
    \replyitem{404}{If the garbage tag does not exist or the index is invalid.}
  \end{datalist}
\end{apidata}

To get a thumbnail, call:

\apilocation{get}{ws/garbage/\param{garbageId}/image/\param{image\_index}/thumb}
\begin{apidata}{returns}
  \begin{datalist}
    \replyditem{An ``image/jpeg'' file containing the picture.}
    \replyitem{404}{If the garbage tag does not exist or the index is invalid.}
  \end{datalist}
\end{apidata}


\subsection{Deleting an image of the garbage tag}

This call deletes an image associated to a garbage. All garbage tags have an
array of images, and new images are pushed to the tail of the array. The array
length can be obtained through the \texttt{pictureCount} field of the
\texttt{garbage} entity. 

To make this call, a valid image index must be known. Valid indexes are from 0
to \texttt{pictureCount} - 1.

\apilocation{delete}{ws/garbage/\param{garbageId}/image/\param{image\_index}}
\begin{apidata}{returns}
  \begin{datalist}
    \replyitem{200}{If the operation succeeds.}
    \replyitem{404}{If the garbage tag does not exist or the index is invalid.}
  \end{datalist}
\end{apidata}
\begin{apidata}{access}
The logged in user must be an ``ADMIN'' or the garbage tag's owner.
\end{apidata}


\subsection{Searching for garbages in a given town}

\apilocation{get}{ws/garbage/townSearch/?town=\param{town\_name}}
\begin{apidata}{returns}
  \begin{datalist}
    \replyditem{A list of \texttt{garbage} entities of the garbages in the given
    town.}
  \end{datalist}
\end{apidata}


\subsection{Searching for garbages in a given county}

\apilocation{get}{ws/garbage/countySearch/?county=\param{county\_name}}
\begin{apidata}{returns}
  \begin{datalist}
    \replyditem{A list of \texttt{garbage} entities of the garbages in the given
    county.}
  \end{datalist}
\end{apidata}


\subsection{Searching for garbages by status}

\apilocation{get}{ws/garbage/statusSearch/?status=\param{status}}
\begin{apidata}{returns}
  \begin{datalist}
    \replyditem{A list of \texttt{garbage} entities of the garbages having the
    given status.}
  \end{datalist}
\end{apidata}

\subsection{Searching for garbages within a bounding box}

\apilocation{get}{ws/garbage/bbox/?topLeftX=\param{topLeftX}\&\\topLeftY=\param{topLeftY}\&bottomRightX=\param{bottomRightX}\&bottomRightY=\param{bottomRightY}\&\\maxResults=\param{maxResults}}
\begin{apidata}{parameters}
  \texttt{maxResults} can be omitted if no upper bound is desired.
\end{apidata}
\begin{apidata}{returns}
  \begin{datalist}
    \replyditem{HTTP 406: if the query generates more results than the specified
    \texttt{maxResults}}
    \replyditem{A list of \texttt{garbage} entities of the garbages within the
    the provided bounding box.}
  \end{datalist}
\end{apidata}

\subsection{Garbage report}

\apilocation{get}{ws/garbage/report?county=\param{county}\&chartedArea=\param{name}\&\\userId=\param{ID}\&insertDate=\param{date}}
\begin{apidata}{params}
  Omitting a query parameter behaves like a wildcard in the query. All
  parameters can be repeated to obtain the result of a disjunction. Dates will
  be specified in the \texttt{yyyy-MM-dd} format.
\end{apidata}
\begin{apidata}{returns}
  A garbage list matching the query criteria. The list can be formatted using
  XML, JSON, CSV, XLS or XLSX. To this end, you need to setup the appropriate
  MIME types in the HTTP Accept headers, i.e., ``application/json'',
  ``application/xml'', ``text/csv'', ``application/vnd.ms-excel'', respectively
  ``application/vnd.openxmlformats-officedocument.spreadsheetml.sheet''.
\end{apidata}
\begin{apidata}{access}
The logged in user must be an ``ADMIN'', ``ORGANIZER'', ``ORGANIZER\_MULTI''. 
\end{apidata}


