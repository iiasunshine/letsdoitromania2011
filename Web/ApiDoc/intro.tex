\chapter{Introduction}

This document introduces the API of the web service used in the organization of
the ``Let's Do It, Romania!'' 2011 campaign. Clients can interact with the web
service using a series of RESTful calls to register users, authenticate users,
manage teams and organizations, and manage garbage tags. During these calls,
clients pass and / or retrieve a series of entities describing users, teams,
garbage tags, and so on. The entities and their properties are described in
Section~\ref{cha:entities}, and the calls are described in
Section~\ref{cha:calls}. 

Clients can be implemented in any language, depending on the target platform.
For instance:
\begin{itemize}
  \item JavaScript, if the client is desired to be running inside a web
    browser,
  \item Java, if the client is desired to be run as an Android or desktop
    application,
  \item Objective C, if the client is desired to be run as an iPhone
    application.
\end{itemize}

Entities may be serialized as either JSON or XML.

Next, we provide examples on how to invoke a call from Java and JavaScript.

\section{Inserting a garbage tag from Java code}

The following assumes that Jersey\footnote{A Java package dedicated to RESTful
calls, available at \url{http://jersey.java.net/}} has been installed. The Java
implementation of the entities is available in the
\href{http://code.google.com/p/letsdoitromania2011/source/browse/#svn\%2Ftrunk\%2FWeb\%2FService\%2FLDIRBackendEJBClient\%2FejbModule\%2Fro\%2Fldir\%2Fdto}{\texttt{ro.ldir.dto}}
package. To make use of the latter, add to your classpath the
\texttt{LDIRBackendEJBClient.jar} resulting from a build of the web service
source code.

The following is a code snippet listing the steps required to insert a garbage
tag in the system:
\begin{lstlisting}[language=Java]
// Set up the call's URL, username and password.
String location = "http://localhost:8080/LDIRBackend/ws/garbage";
String username = "me@me.com";
String password = "password";

// Set up the garbage entity.
Gargage garbage = new Garbage();
garbage.setX(...);
garbage.setY(...);
garbage.setVolume(...);

// Create a Jersey client.
Client client = Client.create();
WebResource resource = client.resource(location);
Builder builder = resource.header(HttpHeaders.AUTHORIZATION, "Basic " +
    new String(Base64.encode(username + ":" + password), Charset.forName("ASCII")));

// Make the call and pass the garbage entity.
ClientResponse cr = builder.entity(garbage, 
        MediaType.APPLICATION_XML).post(ClientResponse.class); 

assert cr.getStatus() == 200;
\end{lstlisting}

More examples are available in the unit tests implemented in the package
\href{http://code.google.com/p/letsdoitromania2011/source/browse/#svn\%2Ftrunk\%2FWeb\%2FService\%2FLDIRBackendTests\%2Fsrc\%2Fro\%2Fldir\%2Ftests}{\texttt{ro.ldir.tests}}.

\begin{note}
The default Jersey client does not resolve references when fetching entities
from the web service. That is, when fetching an user entity, any referenced
organization, team, and so on will not be initialized, i.e., they will be set to
\texttt{null}. To fetch these, you can either:
\begin{itemize}
  \item make a specialized call, e.g., call
    \texttt{\footnotesize{\baselocation}ws/user/\{userid\}/organizations} to
    fetch the organizations managed by an user, or
  \item use a client produced by the
    \href{http://code.google.com/p/letsdoitromania2011/source/browse/trunk/Web/Service/LDIRBackendEJBClient/ejbModule/ro/ldir/idresolver/LdirClientResolverFactory.java}{LdirClientResolverFactory}
    factory. You should note that these clients resolve \emph{one-level
    indirections only}, e.g., they fetch an user's organization, but they do not
    fetch the teams in which these organizations are enrolled.
\end{itemize}
\end{note}
\section{Inserting a garbage tag from JavaScript code}

The following assumes a JSON serializer is used\footnote{Such as the one
available at \url{http://json.org/js.html}.}.
\begin{lstlisting}[language=JavaScript]


// Set up the call's URL, username and password.
var location = "http://localhost:8080/LDIRBackend/ws/garbage";
var username = "me@me.com";
var password = "password";

// The call object.
var req;

// Constructor for call objects.
function createXMLHttpRequest() {
    try {
        return new XMLHttpRequest();
    } catch (e) {
    }
    try {
        return new ActiveXObject("Msxml2.XMLHTTP");
    } catch (e) {
    }
    return new ActiveXObject("Microsoft.XMLHTTP");
}


// A callback which the browser uses to notify changes in the call's state.
function garbageInsertCallback() {
    if (req.readyState != 4)
        return;
    if (req.status != 200) {
        alert('Unable to insert data: ' + req.status);
        return;
    }
    alert('Data ok!');
}


// Set up the garbage entity.
var garbage = {
    "x" : ...,
    "y" : ...,
    "volume" : ...
};

// Serialize the object.
var garbageJSON = JSON.stringify(garbage);

// Make a client.
req = createXMLHttpRequest();
req.open("post", location, true, username, password);
req.onreadystatechange = garbageInsertCallback;
req.setRequestHeader('Content-Type', 'application/json');

// Make an asynchronous call.
req.send(garbageJSON);
\end{lstlisting}

More examples are available in the
\href{http://code.google.com/p/letsdoitromania2011/source/browse/trunk/Web/Service/LDIRBackendWeb/WebContent/index.html}{web
test client} and the JavaScript files it references.  An AJAX framework to make
these call is however recommended for ease of work.
