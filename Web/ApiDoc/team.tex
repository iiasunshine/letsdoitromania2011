\section{Team management}
\label{sec:call:team}

This section describes the calls that affect the state of the teams managed by
the system. These are implemented in the
\svnweb{ro/ldir/ws/TeamWebService}{ro.ldir.ws.TeamWebService} web resource.

Most of calls of this section are relative to a team identifier. There are two
methods that can be used to obtain team identifiers:
\begin{itemize}
  \item indirectly through the appropriate field (\texttt{organizations}) of the
    \texttt{user} entity. Note that the user can either manage the team, or be a
    team member. How a user can be accessed is described in
    Section~\ref{sec:call:user} and Section~\ref{sec:call:user}, or
  \item searching a team by name. This procedure is described in
    Section~\ref{sec:call:team:name}.
\end{itemize}

The procedure to enroll users and organizations in a team is described in
Section~\ref{sec:call:user}, respectively Section~\ref{sec:call:organization}.
It should be noted that a user (or a organization) can be a member of at most
one team. 

All calls that follow must be authenticated using HTTP headers. The username in
the HTTP headers must match a user's email whose state is not ``PENDING'' or
``SUSPENDED''.


\subsection{Creating a new team}
\label{sec:team:create}

This call inserts a new organization in the system. The team manager for the new
organization will be set to match the authenticated user.

A user whose role is ``VOLUNTEER'' or ``ORGANIZER'' can manage a single team.
Only users whose role is either ``VOLUNTEER\_MULTI'' or ``ORGANIZER\_MULTI'' can
manage multiple teams.

\apilocation{post}{ws/team}
\begin{apidata}{contents}
  A \texttt{team} entity describing the new team.
\end{apidata}
\begin{apidata}{returns}
  \begin{datalist}
    \replyitem{200}{if the operation succeeds.}
    \replyitem{500}{if another error occurs.}
  \end{datalist}
\end{apidata}


\subsection{Getting team information}

\apilocation{get}{ws/team/\param{teamId}}
\begin{apidata}{returns}
  \begin{datalist}
    \replyditem{A \texttt{team} entity describing the team.}
  \end{datalist}
\end{apidata}


\subsection{Deleting a team}

\apilocation{delete}{ws/team/\param{teamId}}
\begin{apidata}{returns}
  \begin{datalist}
    \replyitem{200}{If the operation succeeds.}
  \end{datalist}
\end{apidata}
\begin{apidata}{access}
The logged in user must be an ``ADMIN'' or the team's manager.
\end{apidata}


\subsection{Searching for a team by name}
\label{sec:call:team:name}

\apilocation{get}{ws/team/nameSearch/?teamName=\param{teamName}}
\begin{apidata}{returns}
  \begin{datalist}
    \replyditem{A list of \texttt{team} entities whose name match the searching
    criteria.}
  \end{datalist}
\end{apidata}

\subsection{Adding team equipment}

\apilocation{put}{ws/team/\param{teamId}/\param{equipment\_type}}
\begin{apidata}{contents}
  One of the \texttt{GpsEquipment}, \texttt{CleaningEquipment} or
  \texttt{TransportEquipment} that are to be added inside the team's equipment
  list.

  The \texttt{\emph{equipment\_type}} path parameter must match the type of the
  inserted equipment. It can be one of the following strings: ``gps'',
  ``cleaning'' and ``transport''.
\end{apidata}
\begin{apidata}{returns}
  \begin{datalist}
    \replyitem{200}{If the operation succeeds.}
  \end{datalist}
\end{apidata}
\begin{apidata}{access}
The logged in user must be an ``ADMIN'' or the team's manager.
\end{apidata}

\subsection{Getting the team equipment by type}
\label{sec:team:eq_by_type}

\apilocation{get}{ws/team/\param{teamId}/\param{equipment\_type}}
\begin{apidata}{params}
  The \texttt{\emph{equipment\_type}} path parameter must match the type of the
  intended equipment. It can be one of the following strings: ``gps'',
  ``cleaning'' and ``transport''.
\end{apidata}
\begin{apidata}{returns}
  \begin{datalist}
    \replyditem{A list of equipments matching the specified type.}
    \replyitem{404}{If the team is not found.}
  \end{datalist}
\end{apidata}
\begin{apidata}{access}
The logged in user must be an ``ADMIN'' or the team's manager.
\end{apidata}

\subsection{Getting the number of team equipments}
\label{sec:team:eq_cnt}

\apilocation{get}{ws/team/\param{teamId}/equipmentCount}
\begin{apidata}{returns}
  \begin{datalist}
    \replyditem{A number describing how many equipments this team has.}
    \replyitem{404}{If the team is not found.}
  \end{datalist}
\end{apidata}
\begin{apidata}{access}
The logged in user must be an ``ADMIN'', the team's manager or a team member.
\end{apidata}

\subsection{Getting team equipment by index}
\label{sec:team:eq_by_idx}

Given a team equipment index, this call gets information about the equipment.
Valid indexes are in the range $0 \ldots n$, where $n$ is obtained by the call
described in Section~\ref{sec:team:eq_cnt}.

\apilocation{get}{ws/team/\param{teamId}/equipment/\param{equipmentId}}
\begin{apidata}{returns}
  \begin{datalist}
    \replyditem{A \texttt{GpsEquipment}, \texttt{CleaningEquipment} or
    \texttt{TransportEquipment} entity describing the equipment.}
    \replyitem{404}{If the team or the equipment is not found.}
  \end{datalist}
\end{apidata}
\begin{apidata}{access}
The logged in user must be an ``ADMIN'', the team's manager or a team member.
\end{apidata}


\subsection{Removing team equipment by index}

Given a team equipment index, this call deletes the equipment.
Valid indexes are in the range $0 \ldots n$, where $n$ is obtained by the call
described in Section~\ref{sec:team:eq_cnt}.

\apilocation{delete}{ws/team/\param{teamId}/equipment/\param{equipmentId}}
\begin{apidata}{returns}
  \begin{datalist}
    \replyitem{404}{If the team or the equipment is not found.}
  \end{datalist}
\end{apidata}
\begin{apidata}{access}
The logged in user must be an ``ADMIN'', the team's manager or a team member.
\end{apidata}


\subsection{Removing team equipment by ID}

Given an equipment identifier, this call removes the equipment from a team's
ownership. The equipment identifiers are a property of the \texttt{equipment}
entity (i.e., \texttt{equipmentId}). The latter must be obtained through one of
the calls described in Section~\ref{sec:team:eq_by_type} or
Section~\ref{sec:team:eq_by_idx}.

\apilocation{delete}{ws/team/\param{teamId}/equipmentId/\param{equipmentId}}
\begin{apidata}{returns}
  \begin{datalist}
    \replyitem{200}{If the operation succeeds.}
  \end{datalist}
\end{apidata}
\begin{apidata}{access}
The logged in user must be an ``ADMIN'' or the team's manager.
\end{apidata}


\subsection{Withdrawing a user from the member list}

\apilocation{delete}{ws/team/\param{teamId}/volunteer/\param{volunteerId}}
\begin{apidata}{returns}
  \begin{datalist}
    \replyitem{200}{If the operation succeeds.}
  \end{datalist}
\end{apidata}
\begin{apidata}{access}
The logged in user must be an ``ADMIN'' or the team's manager.
\end{apidata}



\subsection{Withdrawing an organization from the member list}

\apilocation{delete}{ws/team/\param{teamId}/organization/\param{volunteerId}}
\begin{apidata}{returns}
  \begin{datalist}
    \replyitem{200}{If the operation succeeds.}
  \end{datalist}
\end{apidata}
\begin{apidata}{access}
The logged in user must be an ``ADMIN'' or the team's manager.
\end{apidata}


\subsection{Assigning a new area to chart}

\apilocation{post}{ws/team/\param{teamId}/chartArea}
\begin{apidata}{contents}
  A \texttt{chartArea} entity whose only set field is the chart area ID.
\end{apidata}
\begin{apidata}{returns}
  \begin{datalist}
    \replyitem{200}{If the operation succeeds.}
    \replyitem{406}{An a string if constraints are violated and the assign
    operations is not accepted.}
  \end{datalist}
\end{apidata}
\begin{apidata}{access}
The logged in user must be an ``ADMIN'' or the team's manager.
\end{apidata}


\subsection{Removing an area from the list of areas to chart}

\apilocation{delete}{ws/team/\param{teamId}/chartArea/\param{chartAreaId}}
\begin{apidata}{returns}
  \begin{datalist}
    \replyitem{200}{If the operation succeeds.}
  \end{datalist}
\end{apidata}
\begin{apidata}{access}
The logged in user must be an ``ADMIN'' or the team's manager.
\end{apidata}


\subsection{Assigning a garbage to be cleaned}

\apilocation{put}{ws/team/\param{teamId}/cleaningGarbages/\param{garbageId}}
\begin{apidata}{contents}
  An integer denoting the number of bags allocated for this garbage.
\end{apidata}
\begin{apidata}{returns}
  \begin{datalist}
    \replyitem{200}{If the operation succeeds.}
    \replyitem{404}{If one of the team or garbage are not found.}
    \replyitem{406}{If the user is trying to allocated too many bags to this
    garbage than it has available, or if the garbage requires a smaller amount
    of bags. An accompanying text will be returned to specified the exact cause
    of error.}
    \replyitem{415}{If the posted number of allocated garbages is not a valid
    integer.}
  \end{datalist}
\end{apidata}
\begin{apidata}{access}
The logged in user must be an ``ADMIN'' or the team's manager.
\end{apidata}

\subsection{Removing a garbage to be cleaned}

\apilocation{delete}{ws/team/\param{teamId}/cleaningGarbages}
\begin{apidata}{contents}
    A \texttt{garbage} entity where the only field set is \texttt{garbageId}.
    The garbage ID must match one already existing in the DB.
\end{apidata}
\begin{apidata}{returns}
  \begin{datalist}
    \replyitem{200}{If the operation succeeds.}
    \replyitem{404}{If one of the team or garbage are not found.}
  \end{datalist}
\end{apidata}
\begin{apidata}{access}
The logged in user must be an ``ADMIN'' or the team's manager.
\end{apidata}

\subsection{Getting the list of garbages to be cleaned}

\apilocation{get}{ws/team/\param{teamId}/cleaningGarbages}
\begin{apidata}{returns}
  \begin{datalist}
    \item{An array of \texttt{entities} assigned to this team.}
  \end{datalist}
\end{apidata}
\begin{apidata}{access}
The logged in user must be an ``ADMIN'' or the team's manager.
\end{apidata}


\subsection{Team report}

\apilocation{get}{ws/team/report?county=\param{county}}
\begin{apidata}{params}
  Omitting a query parameter behaves like a wildcard. All parameters can be
  repeated and the query will behave like a disjunction.
\end{apidata}
\begin{apidata}{returns}
  A list of teams matching the query criteria. The list can be formatted using
  XML, JSON, CSV, XLS or XLSX. To this end, you need to setup the appropriate
  MIME types in the HTTP Accept headers, i.e., ``application/json'',
  ``application/xml'', ``text/csv'', ``application/vnd.ms-excel'', respectively
  ``application/vnd.openxmlformats-officedocument.spreadsheetml.sheet''.
\end{apidata}
\begin{apidata}{access}
The logged in user must be an ``ADMIN'', ``ORGANIZER'', ``ORGANIZER\_MULTI''. 
\end{apidata}

\subsection{Assigned charted area report}

\apilocation{get}{ws/team/reportChartedArea?county=\param{county}\&\\chartedArea=\param{chartedAreaName}\&managerId=\param{managerId}}
\begin{apidata}{params}
  Omitting a query parameter behaves like a wildcard. All parameters can be
  repeated and the query will behave like a disjunction.
\end{apidata}
\begin{apidata}{returns}
  A list of teams matching the query criteria. The list can be formatted using
  XML, JSON, CSV, XLS or XLSX. To this end, you need to setup the appropriate
  MIME types in the HTTP Accept headers, i.e., ``application/json'',
  ``application/xml'', ``text/csv'', ``application/vnd.ms-excel'', respectively
  ``application/vnd.openxmlformats-officedocument.spreadsheetml.sheet''.

  The \texttt{chartedArea} objects assigned to each team in the result can be
  further filtered to match the query criteria using the
  \texttt{ro.ldir.dto.helper.AssignedChartedAreaFilter}. 
\end{apidata}
\begin{apidata}{access}
The logged in user must be an ``ADMIN'', ``ORGANIZER'', ``ORGANIZER\_MULTI''. 
\end{apidata}


