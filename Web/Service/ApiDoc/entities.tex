\chapter{Entities}
\label{cha:entities}

This chapter describes the entities used to interact with the Let's Do It
Romania web service. The calls specified later in Chapter~\ref{cha:calls} accept
as parameters or return one or more of the entities that follow. During the
aforementioned calls, these entities can be serialized either as JSON or XML.

It should be noted that although most of these entities are persisted in the
database:
\begin{itemize}
  \item some of the properties stored in the database are not serialized,
  \item some properties are available only at serialization time and are not
    persisted in the database,
  \item some properties are read only, e.g., the ID of a user which is
    automatically assigned by the system, and
  \item some properties require a special API call to change and cannot be
    changed when an entity is updated. For instance, the user update call
    (described later in Section~\ref{sec:user:update}) leaves unchanged the list
    of teams managed by the user being updated; to update the list of managed
    teams, the user must make a create team call
    (Section~\ref{sec:team:create}). 
\end{itemize}

First (Section~\ref{sec:entities:user}) we present user-related entities, such
as system users, teams and organizations, and then
(Section~\ref{sec:entities:geographical}) we present geographical-related
entities, such as garbage tags, charting areas, and administrative areas.

\section{User-related entities}
\label{sec:entities:user}

\subsection{System users}
The following lists all properties of the \texttt{user} entity. All read-only
properties are immutable with respect to the user update call described in
Section~\ref{sec:user:update}.
\begin{entitydef}{user}
  \roproperty{activities}{A set of string describing the activities that the
  user is enrolled in. Valid activities are ``CHART'' and ``CLEAN''.}
  \property{birthday}{A string containing the user's birthday, formatted as
  ``YYYY-MM-DD''.}
  \property{county}{A string containing the name of the county where the user is
  located.}
  \roproperty{email}{A string containing the user's email address. This property
  is used to log in the system and cannot be modified.}
  \property{firstName}{A string denoting the user's first name.}
  \roproperty{garbages}{A set of integers containing the IDs of the
  \texttt{garbage} tag entities the user has inserted.}
  \property{lastName}{A string denoting the user's last name.}
  \roproperty{managedTeams}{A set of integers containing the IDs of the
  \texttt{team} entities the user manages. }
  \roproperty{memberOf}{An integer denoting the ID of the \texttt{team} entity
  the user is enrolled in.}
  \roproperty{organization}{A set of integers containing the IDs of the
  \texttt{organization} entities the user is responsible for.}
  \property{password}{A string denoting the user's password. When polling the
  system, the web service will return the password encrypted using SHA-256. When
  updating the user, the new password must be provided in clear text.}
  \property{phone}{A string denoting the user's phone number.}
  \roproperty{role}{A string denoting the user's security access role. Valid
  roles are:
  \begin{compactitem}
    \item ``PENDING'': the user has just registered on the website, but the
      email is not validated yet. The user does not have access to the service.
    \item ``SUSPENDED'': the user does not have access to the service.
    \item ``ADMIN'': an administration user.
    \item ``VOLUNTEER'': a volunteer user that can manage only one team.
    \item ``VOLUNTEER\_MULTI'': a volunteer user that can manage multiple teams.
    \item ``ORGANIZER'': a organizer user that can manage only one team.
    \item ``ORGANIZER\_MULTI'': a organizer user that can manage multiple
      teams.
  \end{compactitem} 
}
  \property{town}{A string containing the name of the town where the user is
  located.}
  \roproperty{userId}{An integer denoting the ID of the \texttt{user} entity
  instance.}
\end{entitydef}

\subsection{Organizing users}

\begin{entitydef}{organization}
  \property{address}{A string containing the address of the organization.}
  \roproperty{contactUser}{An integer denoting the ID of the \texttt{user}
  entity responsible for this organization.}
  \property{county}{A string denoting the county where this organization is
  located.}
  \property{memberOf}{An integer denoting the ID of the \texttt{team} entity
  in which this organization is enrolled.}
  \property{membersCount}{An integer denoting the number of members of this
  organization.}
  \roproperty{organizationId}{An integer denoting the ID of the
  \texttt{organization} entity instance.}
  \property{town}{A string denoting the town where this organization is
  located.}
  \property{type}{A string denoting the type of the organization. Valid types
  are: ``CITY\_HALL'', ``COMPANY'', ``LANDFILL'', ``REGISTRATION\_POINT'',
  and ``SCHOOL''.}
\end{entitydef}

\begin{entitydef}{team}
  \roproperty{chartedAreas}{A set of integers denoting the IDs of the
  \texttt{chartArea} entities this team instance is charting.}
  \roproperty{equipment}{A set of equipment entities that this team possesses.
  Valid equipment are: \texttt{cleaningEquipment}, \texttt{gpsEquipment} and
  \texttt{transportEquipment}.}
  \roproperty{garbages}{A set of integers denoting the IDs of the
  \texttt{garbage} entities this team desires to clean.}
  \roproperty{organizationMembers}{A set of integers denoting ID of the
  \texttt{organization} entities enrolled in this team.}
  \roproperty{teamId}{An integer denoting ID of the \texttt{team} entity
  instance.}
  \roproperty{teamManager}{An integer denoting ID of the \texttt{user} entity
  who is managing this team.}
  \roproperty{volunteerMembers}{A set of integers denoting the IDs of the
  \texttt{user} entities enrolled in this team.}
\end{entitydef}

\subsection{Equipments}

The following entities represent objects that a team may use and own during the
charting and cleaning process. All are derived from an abstract entity called
\texttt{equipment}.

\begin{entitydef}{cleaningEquipment}
  \roproperty{equipmentId}{An integer denoting the ID of the equipment entity
  instance.}
  \roproperty{teamOwner}{An integer denoting the ID of the \texttt{team} entity
  owning this equipment.}
  \property{cleaningType}{A string denoting the type of this cleaning equipment.
  Valid types are ``BAGS'' and ``GLOVES.''.}
\end{entitydef}


\begin{entitydef}{gpsEquipment}
  \roproperty{equipmentId}{An integer denoting the ID of the equipment entity
  instance.}
  \roproperty{teamOwner}{An integer denoting the ID of the \texttt{team} entity
  owning this equipment.}
  \property{count}{An integer denoting how many equipments of this type are
  owned by the team.}
\end{entitydef}


\begin{entitydef}{transportEquipment}
  \roproperty{equipmentId}{An integer denoting the ID of the equipment entity
  instance.}
  \roproperty{teamOwner}{An integer denoting the ID of the \texttt{team} entity
  owning this equipment.}
  \property{transportType}{A string denoting the type of this transport
  equipment. Valid transports are ``BICYCLE'', ``CAR'', ``ORGANIZATION\_CAR''
  and ``PUBLIC''}
\end{entitydef}

\section{Geographical entities}
\label{sec:entities:geographical}

\subsection{Garbage tags}

\begin{entitydef}{garbage}
  \roproperty{chartedArea}{An integer denoting the ID of the
  \texttt{chartedArea} entity in which this garbage has been tagged. This field
  is automatically filled in by the system based on the geographical coordinates
  of the garbage tag at the moment when the tag is inserted in the system.}
  \roproperty{county}{An integer denoting the ID of the \texttt{countyArea}
  entity in which this garbage has been tagged. This field is automatically
  filled in by the system based on the geographical coordinates of the garbage
  tag at the moment when the tag is inserted in the system.}
  \property{description}{A user-inserted string.}
  \roproperty{enrolledCleaners}{A list of integers denoting the ID of the
  \texttt{team} entities that wish to clean this garbage.}
  \roproperty{garbageId}{An integer denoting the ID of this garbage tag entity
  instance.}
  \roproperty{pictures}{An array of strings, each denoting the location of a
  picture illustrating the tagged garbage.}
  \property{status}{A string denoting the garbage tag's status. Can be
  ``CLEANED'' or ``IDENTIFIED''.}
  \roproperty{town}{An integer denoting the ID of the \texttt{townArea}
  entity in which this garbage has been tagged. This field is automatically
  filled in by the system based on the geographical coordinates of the garbage
  tag at the moment when the tag is inserted in the system.}
  \property{volume}{A user-inserted integer.}
  \property{x}{A float denoting the garbage tag's longitude.}
  \property{y}{A float denoting the garbage tag's latitude.}
\end{entitydef}

\subsection{Areas}


All of the following tags are derived from a base abstract entity called
\texttt{closedArea}. Such an entity denotes a (closed) polygon made up by
vertexes expressed as geographical coordinates.

\begin{entitydef}{chartedArea}
  \roproperty{areaId}{An integer denoting the ID of the \texttt{chartedArea}
  entity instance.}
  \property{polyline}{An ordered list of float pairs, each denoting a
  geographical coordinates. Consecutive entries in this list make the vertexes
  of a side of the polygon. It is assumed that the last entry and the first
  entry make up for one polygon sides.}
  \roproperty{topLeftX}{A float denoting the longitude of a vertex which is the
  top left corner of the \texttt{chartedArea}'s bounding box.}
  \roproperty{topLeftY}{A float denoting the latitude of a vertex which is the
  top left corner of the \texttt{chartedArea}'s bounding box.}
  \roproperty{bottomRightX}{A float denoting the longitude of a vertex which is
  the bottom Right corner of the \texttt{chartedArea}'s bounding box.}
  \roproperty{bottomRightY}{A float denoting the latitude of a vertex which is
  the bottom Right corner of the \texttt{chartedArea}'s bounding box.}
  \roproperty{chartedArea}{A list of integers denoting the IDs of the teams
  charting this area.}
  \roproperty{garbages}{A list of integers denoting the IDs of the garbage tags
  inserted in this area.}
  \property{score}{An integer denoting the score of the charted area.}
\end{entitydef}

\begin{entitydef}{countyArea}
  \roproperty{areaId}{An integer denoting the ID of the \texttt{countyArea}
  entity instance.}
  \property{polyline}{An ordered list of float pairs, each denoting a
  geographical coordinates. Consecutive entries in this list make the vertexes
  of a side of the polygon. It is assumed that the last entry and the first
  entry make up for one polygon sides.}
  \roproperty{topLeftX}{A float denoting the longitude of a vertex which is the
  top left corner of the \texttt{countyArea}'s bounding box.}
  \roproperty{topLeftY}{A float denoting the latitude of a vertex which is the
  top left corner of the \texttt{countyArea}'s bounding box.}
  \roproperty{bottomRightX}{A float denoting the longitude of a vertex which is
  the bottom Right corner of the \texttt{countyArea}'s bounding box.}
  \roproperty{bottomRightY}{A float denoting the latitude of a vertex which is
  the bottom Right corner of the \texttt{countyArea}'s bounding box.}
  \roproperty{countyArea}{A list of integers denoting the IDs of the teams
  charting this area.}
  \roproperty{garbages}{A list of integers denoting the IDs of the garbage tags
  inserted in this area.}
  \property{name}{A string denoting the name of the county area.}
\end{entitydef}

\begin{entitydef}{townArea}
  \roproperty{areaId}{An integer denoting the ID of the \texttt{townArea}
  entity instance.}
  \property{polyline}{An ordered list of float pairs, each denoting a
  geographical coordinates. Consecutive entries in this list make the vertexes
  of a side of the polygon. It is assumed that the last entry and the first
  entry make up for one polygon sides.}
  \roproperty{topLeftX}{A float denoting the longitude of a vertex which is the
  top left corner of the \texttt{townArea}'s bounding box.}
  \roproperty{topLeftY}{A float denoting the latitude of a vertex which is the
  top left corner of the \texttt{townArea}'s bounding box.}
  \roproperty{bottomRightX}{A float denoting the longitude of a vertex which is
  the bottom Right corner of the \texttt{townArea}'s bounding box.}
  \roproperty{bottomRightY}{A float denoting the latitude of a vertex which is
  the bottom Right corner of the \texttt{townArea}'s bounding box.}
  \roproperty{townArea}{A list of integers denoting the IDs of the teams
  charting this area.}
  \roproperty{garbages}{A list of integers denoting the IDs of the garbage tags
  inserted in this area.}
  \property{name}{A string denoting the name of the town area.}
\end{entitydef}
