\section{User management}
\label{sec:call:user}

This section describes the calls that affect the state of a system user. These
are implemented in the
\svnweb{ro/ldir/ws/UserWebService}{ro.ldir.ws.UserWebService} web resource.

It should be noted that before making most of the calls in this section, the
user ID of the user being accessed must be known. Section~\ref{sec:user:crt}
shows how to obtain the logged-in user's ID.

It should be noted that there are some security considerations that must be
taken into account when making some of the calls that follows. For instance, a
non-admin user should be allowed to retrieve and update his own state, but he
should not be able to change his security role or change the state of another
user. At the same time, an admin should be able to have complete access to all
users.

All calls that follow must be authenticated using HTTP headers. The username in
the HTTP headers must match a user's email whose state is not ``PENDING'' or
``SUSPENDED''.


\subsection{Obtaining the logged-in user's ID}
\label{sec:user:crt}

\apilocation{get}{ws/user}
\begin{apidata}{returns}
  \begin{datalist}
    \replyditem{a string}{containing the ID of the authenticated user.}
    \replyitem{403}{if the supplied credentials do not match a valid user.}
  \end{datalist}
\end{apidata}


\subsection{Getting user information}

\apilocation{get}{ws/user/\param{userId}}
\begin{apidata}{returns}
  \begin{datalist}
    \replyditem{user}{A \texttt{user} entity describing the user.}
    \replyitem{403}{if the access policy if violated.}
    \replyitem{404}{if the user is not found.}
  \end{datalist}
\end{apidata}
\begin{apidata}{access}
The logged in user must be an ``ADMIN'', the user being accessed, or a user from
a team in which the accessed user is either part of or manager.
\end{apidata}


\subsection{Updating user information}
\label{sec:user:update}

\apilocation{put}{ws/user/\param{userId}}
\begin{apidata}{content}
  A \texttt{user} entity containing update information.
\end{apidata}
\begin{apidata}{returns}
  \begin{datalist}
    \replyitem{200}{if the operation succeeds.}
    \replyitem{403}{if the access policy if violated.}
    \replyitem{404}{if the user does not exists.}
    \replyitem{500}{if another error occurs.}
  \end{datalist}
\end{apidata}
\begin{apidata}{access}
The logged in user must be an ``ADMIN'' or the user being changed.
\end{apidata}


\subsection{Obtaining user's activities}

\apilocation{get}{ws/user/\param{userId}/activities}
\begin{apidata}{returns}
  \begin{datalist}
    \replyditem{the activity list of the user.}
    \replyitem{403}{if the access policy if violated.}
    \replyitem{404}{if the user does not exists.}
    \replyitem{500}{if another error occurs.}
  \end{datalist}
\end{apidata}
\begin{apidata}{access}
The logged in user must be an ``ADMIN'', the user being accessed, or a user from
a team in which the accessed user is either part of or manager.
\end{apidata}


\subsection{Enrolling a user in a team}

\apilocation{post}{ws/user/\param{userId}/team}
\begin{apidata}{content}
  A \texttt{team} entity where only the \texttt{teamId} field is set.
\end{apidata}
\begin{apidata}{returns}
  \begin{datalist}
    \replyitem{200}{if the operation succeeds.}
    \replyitem{404}{if the user does not exists.}
    \replyitem{409}{if the user is already enrolled in a team.}
    \replyitem{500}{if another error occurs.}
  \end{datalist}
\end{apidata}
\begin{apidata}{access}
The logged in user must be an ``ADMIN'' or the user being changed.
\end{apidata}


\subsection{Updating user activities}

\apilocation{put}{ws/user/\param{userId}/team}
\begin{apidata}{content}
  A list of strings defining the new activities that the user is enrolled for.
\end{apidata}
\begin{apidata}{returns}
  \begin{datalist}
    \replyitem{200}{if the operation succeeds.}
    \replyitem{404}{if the user does not exists.}
    \replyitem{500}{if another error occurs.}
  \end{datalist}
\end{apidata}
\begin{apidata}{access}
The logged in user must be an ``ADMIN'' or the user being changed.
\end{apidata}


\subsection{Setting user security roles}
\apilocation{put}{ws/user/\param{userId}/role}
\begin{apidata}{content}
  A string defining the new security role of the user. 
\end{apidata}
\begin{apidata}{returns}
  \begin{datalist}
    \replyitem{200}{if the operation succeeds.}
    \replyitem{404}{if the user does not exists.}
    \replyitem{500}{if another error occurs.}
  \end{datalist}
\end{apidata}
\begin{apidata}{access}
The logged in user must be an ``ADMIN''.
\end{apidata}

\subsection{User report}

\apilocation{get}{ws/user/report?county=\param{county}\&birthyear=\param{year}\&\\role=\param{role}\&minGarbages=\param{min}\&maxGarbages=\param{max}}
\begin{apidata}{params}
  Omitting a query parameter behaves like a wildcard in the query. The
  \texttt{county}, \texttt{year} and \texttt{role} parameters can be repeated to
  obtain the result oh a disjunction.
\end{apidata}
\begin{apidata}{returns}
  A user list matching the query criteria. The list can be formatted using XML,
  JSON, CSV, XLS or XLSX. To this end, you need to setup the appropriate MIME
  types in the HTTP Accept headers, i.e., ``application/json'',
  ``application/xml'', ``text/csv'', ``application/vnd.ms-excel'', respectively
  ``application/vnd.openxmlformats-officedocument.spreadsheetml.sheet''.
\end{apidata}
\begin{apidata}{access}
The logged in user must be an ``ADMIN'', ``ORGANIZER'', ``ORGANIZER\_MULTI''. 
\end{apidata}


