\section{Garbage tag management}

This section describes the calls that affect the state of garbage tags inserted
in the system. These are implemented in the
\svnweb{ro/ldir/ws/GarbageWebService}{ro.ldir.ws.GarbageWebService} web
resource.

All calls to this web service must be authenticated using HTTP headers. The
username in the HTTP headers must match a user's email whose security role is
``ADMIN''.


\subsection{Adding a new garbage tag}

This call inserts a new garbage in the system. The owner of the garbage will be
set to match the authenticated user.

The town, county and charting area of the new garbage tag will be set
automatically by the system using the tag's coordinates and the entities stored
in the database.

\apilocation{post}{ws/garbage}
\begin{apidata}{contents}
  A \texttt{garbage} entity describing the new garbage tag.
\end{apidata}
\begin{apidata}{returns}
  \begin{datalist}
    \replyitem{200}{if the operation succeeds.}
    \replyitem{400}{if the garbage coordinates are wrong, i.e., they do not
    belong to any county.}
    \replyitem{500}{if another error occurs.}
  \end{datalist}
\end{apidata}


\subsection{Getting garbage tag information}

\apilocation{get}{ws/garbage/\param{garbageId}}
\begin{apidata}{returns}
  \begin{datalist}
    \replyditem{A \texttt{garbage} entity describing the garbage tag.}
  \end{datalist}
\end{apidata}

\subsection{Updating a garbage tag}

\apilocation{post}{ws/garbage/\param{garbageId}}
\begin{apidata}{contents}
  A \texttt{garbage} entity describing the new garbage tag.
\end{apidata}
\begin{apidata}{returns}
  \begin{datalist}
    \replyitem{200}{if the operation succeeds.}
    \replyitem{400}{if the garbage coordinates are wrong, i.e., they do not
    belong to any county.}
    \replyitem{500}{if another error occurs.}
  \end{datalist}
\end{apidata}


% \subsection{Deleting a garbage tag}
% 
% This method is not implemented yet.
% 
% \apilocation{delete}{ws/garbage/\param{garbageId}}
% \begin{apidata}{returns}
%   \begin{datalist}
%     \replyitem{200}{If the operation succeeds.}
%   \end{datalist}
% \end{apidata}
% \begin{apidata}{access}
% The logged in user must be an ``ADMIN'' or the garbage tag's owner.
% \end{apidata}


\subsection{Setting the status of a garbage tag}

\apilocation{put}{ws/garbage/\param{garbageId}/status}
\begin{apidata}{contents}
  A string denoting the new status of the garbage. Can be either ``CLEANED'' or
  ``IDENTIFIED''.
\end{apidata}
\begin{apidata}{returns}
  \begin{datalist}
    \replyitem{200}{if the operation succeeds.}
    \replyitem{500}{if another error occurs.}
  \end{datalist}
\end{apidata}
\begin{apidata}{access}
The logged in user must be an ``ADMIN'' or the garbage tag's owner.
\end{apidata}


\subsection{Assigning a new image to a garbage tag}

\apilocation{post}{ws/garbage/\param{garbageId}/image}
\begin{apidata}{contents}
  A ``multipart/form-data'' request containing a field ``file'' representing a
  new image to be associated to this garbage tag.
\end{apidata}
\begin{apidata}{returns}
  \begin{datalist}
    \replyitem{200}{if the operation succeeds.}
    \replyitem{404}{if the garbage tag does not exist.}
    \replyitem{500}{if another error occurs.}
  \end{datalist}
\end{apidata}
\begin{apidata}{access}
The logged in user must be an ``ADMIN'' or the garbage tag's owner.
\end{apidata}


\subsection{Getting an image of the garbage tag}

This call retrieves an image associated to a garbage. All garbage tags have an
array of images, and new images are pushed to the tail of the array. The array
length can be obtained through the \texttt{pictureCount} field of the
\texttt{garbage} entity. 

To make this call, a valid image index must be known. Valid indexes are from 0
to \texttt{pictureCount} - 1.

\apilocation{get}{ws/garbage/\param{garbageId}/image/\param{image\_index}}
\begin{apidata}{returns}
  \begin{datalist}
    \replyditem{An ``image/jpeg'' file containing the picture.}
    \replyitem{404}{If the garbage tag does not exist or the index is invalid.}
  \end{datalist}
\end{apidata}


\subsection{Deleting an image of the garbage tag}

This call deletes an image associated to a garbage. All garbage tags have an
array of images, and new images are pushed to the tail of the array. The array
length can be obtained through the \texttt{pictureCount} field of the
\texttt{garbage} entity. 

To make this call, a valid image index must be known. Valid indexes are from 0
to \texttt{pictureCount} - 1.

\apilocation{delete}{ws/garbage/\param{garbageId}/image/\param{image\_index}}
\begin{apidata}{returns}
  \begin{datalist}
    \replyitem{200}{If the operation succeeds.}
    \replyitem{404}{If the garbage tag does not exist or the index is invalid.}
  \end{datalist}
\end{apidata}
\begin{apidata}{access}
The logged in user must be an ``ADMIN'' or the garbage tag's owner.
\end{apidata}


\subsection{Searching for garbages in a given town}

\apilocation{get}{ws/garbage/townSearch/?town=\param{town\_name}}
\begin{apidata}{returns}
  \begin{datalist}
    \replyditem{A list of \texttt{garbage} entities of the garbages in the given
    town.}
  \end{datalist}
\end{apidata}


\subsection{Searching for garbages in a given county}

\apilocation{get}{ws/garbage/countySearch/?county=\param{county\_name}}
\begin{apidata}{returns}
  \begin{datalist}
    \replyditem{A list of \texttt{garbage} entities of the garbages in the given
    county.}
  \end{datalist}
\end{apidata}


\subsection{Searching for garbages by status}

\apilocation{get}{ws/garbage/statusSearch/?status=\param{status}}
\begin{apidata}{returns}
  \begin{datalist}
    \replyditem{A list of \texttt{garbage} entities of the garbages having the
    given status.}
  \end{datalist}
\end{apidata}

\subsection{Searching for garbages within a bounding box}

\apilocation{get}{ws/garbage/bbox/?topLeftX=\param{topLeftX}\&\\topLeftY=\param{topLeftY}\&bottomRightX=\param{bottomRightX}\&bottomRightY=\param{bottomRightY}}
\begin{apidata}{returns}
  \begin{datalist}
    \replyditem{A list of \texttt{garbage} entities of the garbages within the
    the provided bounding box.}
  \end{datalist}
\end{apidata}
