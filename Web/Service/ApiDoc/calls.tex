\chapter{Calls}
\label{cha:calls}

This chapter describes the calls that the Let's Do It Romania Web Service
accepts. The assumption is that the web service runs on the local machine and it
is accessible at \baselocation. If the web service runs at a different location,
the string ``\baselocation'' must be properly set for each call.

Calls in this section accept take as arguments or return entities described in
Section~\ref{cha:entities} serialized as JSON or XML. The preferred format is
specified in the HTTP headers. For instance, if a client desired the web service
to output format in JSON format during a GET call, the HTTP headers must include
``\texttt{accepts: application/json}''. XML serialization can be obtained by
using the header ``\texttt{accepts: application/xml}''. When an entity is
uploaded to the web service, for instance, through a POST call, the headers to
be used are ``\texttt{content: application/json}'' if the uploaded entity is
provided in JSON format, respectively ``\texttt{content: application/xml}'' when
XML is being employed.

Hereafter, if:
\begin{itemize}
  \item {\sffamily \bfseries method} is a HTTP method such as get, post, put,
    delete, and
  \item \texttt{\param{url}} is the location of a call,
\end{itemize}
then the syntax of a call is represented as:

\bigskip

\apilocation{METHOD}{\param{url}}
\begin{apidata}{content}
  Valid only for put and post requests, describes what kind of data the call
  expects to find embedded in the request.
\end{apidata}
\begin{apidata}{returns}
  Describes a list of possible return values of the call. It can either be an
  entity, a HTTP result code or raw content.
\end{apidata}
\begin{apidata}{access}
  A brief description of who is allowed to make this call.
\end{apidata}

\section{User registration}

The registration of new users is handled by the
\svnweb{ro/ldir/registration/RegistrationWebService}{ro.ldir.registration.RegistrationWebService}
web resource. Registration is a two step process. In a first step, the new user
submits its details to the system. The system will mark the user's status as
``PENDING''. Then, the system will send the new user an email containing a token
used to validate the user's email address. In the second step, when the user
validates the email, the system moves the user in the default role
``VOLUNTEER''. Later, an administrator can change the role of the recently
validated user. The latter procedure is covered in Section~\ref{sec:call:user}.

All calls described throughout this section are sent without authentication.

\subsection{Submitting new user details}

\apilocation{post}{reg/ws}
\begin{apidata}{content}
  A \texttt{user} entity containing the user to register.
\end{apidata}
\begin{apidata}{returns}
  \begin{datalist}
    \replyitem{200}{if the operation succeeds.}
    \replyitem{409}{if the email is already in use.}
    \replyitem{500}{if another error occurs.}
  \end{datalist}
\end{apidata}

\subsection{Validating the user's email}
\apilocation{get}{reg/ws/\param{userId}/\param{key}}
\begin{apidata}{returns}
  \begin{datalist}
    \replyitem{200}{if the operation succeeds.}
    \replyitem{404}{if user does not exist.}
    \replyitem{409}{if the token is incorrect or the user is already validated.}
  \end{datalist}
\end{apidata}

\subsection{Resetting user passwords}

To reset the password of a user, you must first generate a token using the call:

\apilocation{post}{reg/ws/reset?email=\param{email}}
\begin{apidata}{returns}
  \begin{datalist}
    \replyitem{200}{if the operation succeeds.}
    \replyitem{404}{if user does not exist.}
    \replyitem{500}{if another error occurs.}
  \end{datalist}
\end{apidata}

The previous call sends a token (valid for a limited amount of time) to the user
via email. The token must be used in the following call along which you must
pass the new password:

\apilocation{post}{reg/ws/reset/\param{userId}/\param{token}}
\begin{apidata}{content}
  A string containing the new password of the user.
\end{apidata}
\begin{apidata}{returns}
  \begin{datalist}
    \replyitem{200}{if the operation succeeds.}
    \replyitem{404}{if user does not exist.}
    \replyitem{406}{if the token is invalid.}
    \replyitem{500}{if another error occurs.}
  \end{datalist}
\end{apidata}




\section{User management}
\label{sec:call:user}

This section describes the calls that affect the state of a system user. These
are implemented in the
\svnweb{ro/ldir/ws/UserWebService}{ro.ldir.ws.UserWebService} web resource.

It should be noted that before making most of the calls in this section, the
user ID of the user being accessed must be known. Section~\ref{sec:user:crt}
shows how to obtain the logged-in user's ID.

It should be noted that there are some security considerations that must be
taken into account when making some of the calls that follows. For instance, a
non-admin user should be allowed to retrieve and update his own state, but he
should not be able to change his security role or change the state of another
user. At the same time, an admin should be able to have complete access to all
users.

All calls that follow must be authenticated using HTTP headers. The username in
the HTTP headers must match a user's email whose state is not ``PENDING'' or
``SUSPENDED''.


\subsection{Obtaining the logged-in user's ID}
\label{sec:user:crt}

\apilocation{get}{ws/user}
\begin{apidata}{returns}
  \begin{datalist}
    \replyditem{a string}{containing the ID of the authenticated user.}
    \replyitem{403}{if the supplied credentials do not match a valid user.}
  \end{datalist}
\end{apidata}


\subsection{Getting user information}

\apilocation{get}{ws/user/\param{userId}}
\begin{apidata}{returns}
  \begin{datalist}
    \replyditem{user}{A \texttt{user} entity describing the user.}
    \replyitem{403}{if the access policy if violated.}
    \replyitem{404}{if the user is not found.}
  \end{datalist}
\end{apidata}
\begin{apidata}{access}
The logged in user must be an ``ADMIN'', the user being accessed, or a user from
a team in which the accessed user is either part of or manager.
\end{apidata}


\subsection{Updating user information}
\label{sec:user:update}

\apilocation{put}{ws/user/\param{userId}}
\begin{apidata}{content}
  A \texttt{user} entity containing update information.
\end{apidata}
\begin{apidata}{returns}
  \begin{datalist}
    \replyitem{200}{if the operation succeeds.}
    \replyitem{403}{if the access policy if violated.}
    \replyitem{404}{if the user does not exists.}
    \replyitem{500}{if another error occurs.}
  \end{datalist}
\end{apidata}
\begin{apidata}{access}
The logged in user must be an ``ADMIN'' or the user being changed.
\end{apidata}


\subsection{Obtaining user's activities}

\apilocation{get}{ws/user/\param{userId}/activities}
\begin{apidata}{returns}
  \begin{datalist}
    \replyditem{the activity list of the user.}
    \replyitem{403}{if the access policy if violated.}
    \replyitem{404}{if the user does not exists.}
    \replyitem{500}{if another error occurs.}
  \end{datalist}
\end{apidata}
\begin{apidata}{access}
The logged in user must be an ``ADMIN'', the user being accessed, or a user from
a team in which the accessed user is either part of or manager.
\end{apidata}


\subsection{Enrolling a user in a team}

\apilocation{post}{ws/user/\param{userId}/team}
\begin{apidata}{content}
  A \texttt{team} entity where only the \texttt{teamId} field is set.
\end{apidata}
\begin{apidata}{returns}
  \begin{datalist}
    \replyitem{200}{if the operation succeeds.}
    \replyitem{404}{if the user does not exists.}
    \replyitem{409}{if the user is already enrolled in a team.}
    \replyitem{500}{if another error occurs.}
  \end{datalist}
\end{apidata}
\begin{apidata}{access}
The logged in user must be an ``ADMIN'' or the user being changed.
\end{apidata}


\subsection{Updating user activities}

\apilocation{put}{ws/user/\param{userId}/team}
\begin{apidata}{content}
  A list of strings defining the new activities that the user is enrolled for.
\end{apidata}
\begin{apidata}{returns}
  \begin{datalist}
    \replyitem{200}{if the operation succeeds.}
    \replyitem{404}{if the user does not exists.}
    \replyitem{500}{if another error occurs.}
  \end{datalist}
\end{apidata}
\begin{apidata}{access}
The logged in user must be an ``ADMIN'' or the user being changed.
\end{apidata}


\subsection{Setting user security roles}
\apilocation{put}{ws/user/\param{userId}/role}
\begin{apidata}{content}
  A string defining the new security role of the user. 
\end{apidata}
\begin{apidata}{returns}
  \begin{datalist}
    \replyitem{200}{if the operation succeeds.}
    \replyitem{404}{if the user does not exists.}
    \replyitem{500}{if another error occurs.}
  \end{datalist}
\end{apidata}
\begin{apidata}{access}
The logged in user must be an ``ADMIN''.
\end{apidata}


\section{Searching users}
\label{sec:call:user_search}

This section describes methods that can be used to search for system users by
providing various searching criteria. These are implemented in the
\svnweb{ro/ldir/ws/UserWebService}{ro.ldir.ws.UserWebService} web resource.

The methods in this section are available for using, but not properly tested and
are likely to experience unexpected behavior.

\subsection{Searching users by enrolled activities}

This method has incomplete implementation.

\apilocation{get}{ws/user/byActivity/\param{activity}}
\begin{apidata}{returns}
  \begin{datalist}
    \replyditem{list of users}{A list of user entities.}
  \end{datalist}
\end{apidata}
\begin{apidata}{access}
  TBD.
\end{apidata}

\subsection{Searching users by town}

This method has incomplete implementation.

\apilocation{get}{ws/user/byTown/?town=\param{town}}
\begin{apidata}{returns}
  \begin{datalist}
    \replyditem{list of users}{A list of user entities.}
  \end{datalist}
\end{apidata}
\begin{apidata}{access}
  TBD.
\end{apidata}

\subsection{Searching users by security role}

This method has incomplete implementation.

\apilocation{get}{ws/user/byRole/\param{role}}
\begin{apidata}{returns}
  \begin{datalist}
    \replyditem{list of users}{A list of user entities.}
  \end{datalist}
\end{apidata}
\begin{apidata}{access}
  TBD.
\end{apidata}

\subsection{Searching users by email}

This method has incomplete implementation.

\apilocation{get}{ws/user/?email=\param{email}}
\begin{apidata}{returns}
  \begin{datalist}
    \replyditem{list of users}{A list of user entities.}
  \end{datalist}
\end{apidata}
\begin{apidata}{access}
  TBD.
\end{apidata}



\section{Organization management}
\label{sec:call:organization}

This section describes the calls that affect the state of organization. These
are implemented in the
\svnweb{ro/ldir/ws/OrganizationWebService}{ro.ldir.ws.OrganizationWebService}
web resource.

Most of calls of this section are relative to an organization identifier. At the
present state, the only way to obtain organization identifiers is indirectly
through the appropriate field (\texttt{organizations}) of the \texttt{user}
entity. How a user can be accessed is described in Section~\ref{sec:call:user}
and Section~\ref{sec:call:user}. 

All calls that follow must be authenticated using HTTP headers. The username in
the HTTP headers must match a user's email whose state is not ``PENDING'' or
``SUSPENDED''.


\subsection{Registering a new organization}

This call inserts a new organization in the system. The contact person for the
new organization will be set to match the authenticated user.

\apilocation{post}{ws/organization}
\begin{apidata}{contents}
  A \texttt{organization} entity describing the new organization.
\end{apidata}
\begin{apidata}{returns}
  \begin{datalist}
    \replyitem{200}{if the operation succeeds.}
    \replyitem{500}{if another error occurs.}
  \end{datalist}
\end{apidata}


\subsection{Getting organization information}

\apilocation{get}{ws/organization/\param{organizationId}}
\begin{apidata}{returns}
  \begin{datalist}
    \replyditem{A \texttt{organization} entity describing the organization.}
  \end{datalist}
\end{apidata}
\begin{apidata}{access}
The logged in user must be an ``ADMIN'', the user who is the organization's
contact, or a user from a team in which the accessed organization's contact is
either part of or manager.
\end{apidata}


\subsection{Deleting an organization}

\apilocation{delete}{ws/organization/\param{organizationId}}
\begin{apidata}{returns}
  \begin{datalist}
    \replyitem{200}{If the operation succeeds.}
  \end{datalist}
\end{apidata}
\begin{apidata}{access}
The logged in user must be an ``ADMIN'' or the organization's contact.
\end{apidata}


\subsection{Updating an organization}

\apilocation{put}{ws/organization/\param{organizationId}}
\begin{apidata}{contents}
  A \texttt{organization} entity containing the new properties of the
  organization.
\end{apidata}
\begin{apidata}{returns}
  \begin{datalist}
    \replyitem{200}{if the operation succeeds.}
    \replyitem{500}{if another error occurs.}
  \end{datalist}
\end{apidata}
\begin{apidata}{access}
The logged in user must be an ``ADMIN'' or the organization's contact.
\end{apidata}


\subsection{Enrolling an organization in a team}

\apilocation{post}{ws/organization/\param{organizationID}/team}
\begin{apidata}{content}
  A \texttt{team} entity where only the \texttt{teamId} field is set.
\end{apidata}
\begin{apidata}{returns}
  \begin{datalist}
    \replyitem{200}{if the operation succeeds.}
    \replyitem{404}{if the organization does not exists.}
    \replyitem{409}{if the organization is already enrolled in a team.}
    \replyitem{500}{if another error occurs.}
  \end{datalist}
\end{apidata}
\begin{apidata}{access}
The logged in user must be an ``ADMIN'' or the organization's contact.
\end{apidata}


\section{Team management}
\label{sec:call:team}

This section describes the calls that affect the state of the teams managed by
the system. These are implemented in the
\svnweb{ro/ldir/ws/TeamWebService}{ro.ldir.ws.TeamWebService} web resource.

Most of calls of this section are relative to a team identifier. There are two
methods that can be used to obtain team identifiers:
\begin{itemize}
  \item indirectly through the appropriate field (\texttt{organizations}) of the
    \texttt{user} entity. Note that the user can either manage the team, or be a
    team member. How a user can be accessed is described in
    Section~\ref{sec:call:user} and Section~\ref{sec:call:user}, or
  \item searching a team by name. This procedure is described in
    Section~\ref{sec:call:team:name}.
\end{itemize}

The procedure to enroll users and organizations in a team is described in
Section~\ref{sec:call:user}, respectively Section~\ref{sec:call:organization}.
It should be noted that a user (or a organization) can be a member of at most
one team. 

All calls that follow must be authenticated using HTTP headers. The username in
the HTTP headers must match a user's email whose state is not ``PENDING'' or
``SUSPENDED''.


\subsection{Creating a new team}
\label{sec:team:create}

This call inserts a new organization in the system. The team manager for the new
organization will be set to match the authenticated user.

A user whose role is ``VOLUNTEER'' or ``ORGANIZER'' can manage a single team.
Only users whose role is either ``VOLUNTEER\_MULTI'' or ``ORGANIZER\_MULTI'' can
manage multiple teams.

\apilocation{post}{ws/team}
\begin{apidata}{contents}
  A \texttt{team} entity describing the new team.
\end{apidata}
\begin{apidata}{returns}
  \begin{datalist}
    \replyitem{200}{if the operation succeeds.}
    \replyitem{500}{if another error occurs.}
  \end{datalist}
\end{apidata}


\subsection{Getting team information}

\apilocation{get}{ws/team/\param{teamId}}
\begin{apidata}{returns}
  \begin{datalist}
    \replyditem{A \texttt{team} entity describing the team.}
  \end{datalist}
\end{apidata}


\subsection{Deleting a team}

\apilocation{delete}{ws/team/\param{teamId}}
\begin{apidata}{returns}
  \begin{datalist}
    \replyitem{200}{If the operation succeeds.}
  \end{datalist}
\end{apidata}
\begin{apidata}{access}
The logged in user must be an ``ADMIN'' or the team's manager.
\end{apidata}


\subsection{Searching for a team by name}
\label{sec:call:team:name}

\apilocation{get}{ws/team/nameSearch/?teamName=\param{teamName}}
\begin{apidata}{returns}
  \begin{datalist}
    \replyditem{A list of \texttt{team} entities whose name match the searching
    criteria.}
  \end{datalist}
\end{apidata}

\subsection{Adding team equipment}

\apilocation{put}{ws/team/\param{teamId}/\param{equipment\_type}}
\begin{apidata}{contents}
  One of the \texttt{GpsEquipment}, \texttt{CleaningEquipment} or
  \texttt{TransportEquipment} that are to be added inside the team's equipment
  list.

  The \texttt{\emph{equipment\_type}} path parameter must match the type of the
  inserted equipment. It can be one of the following strings: ``gps'',
  ``cleaning'' and ``transport''.
\end{apidata}
\begin{apidata}{returns}
  \begin{datalist}
    \replyitem{200}{If the operation succeeds.}
  \end{datalist}
\end{apidata}
\begin{apidata}{access}
The logged in user must be an ``ADMIN'' or the team's manager.
\end{apidata}

\subsection{Getting the team equipment by type}
\label{sec:team:eq_by_type}

\apilocation{get}{ws/team/\param{teamId}/\param{equipment\_type}}
\begin{apidata}{params}
  The \texttt{\emph{equipment\_type}} path parameter must match the type of the
  intended equipment. It can be one of the following strings: ``gps'',
  ``cleaning'' and ``transport''.
\end{apidata}
\begin{apidata}{returns}
  \begin{datalist}
    \replyditem{A list of equipments matching the specified type.}
    \replyitem{404}{If the team is not found.}
  \end{datalist}
\end{apidata}
\begin{apidata}{access}
The logged in user must be an ``ADMIN'' or the team's manager.
\end{apidata}

\subsection{Getting the number of team equipments}
\label{sec:team:eq_cnt}

\apilocation{get}{ws/team/\param{teamId}/equipmentCount}
\begin{apidata}{returns}
  \begin{datalist}
    \replyditem{A number describing how many equipments this team has.}
    \replyitem{404}{If the team is not found.}
  \end{datalist}
\end{apidata}
\begin{apidata}{access}
The logged in user must be an ``ADMIN'', the team's manager or a team member.
\end{apidata}

\subsection{Getting team equipment by index}
\label{sec:team:eq_by_idx}

Given a team equipment index, this call gets information about the equipment.
Valid indexes are in the range $0 \ldots n$, where $n$ is obtained by the call
described in Section~\ref{sec:team:eq_cnt}.

\apilocation{get}{ws/team/\param{teamId}/equipment/\param{equipmentId}}
\begin{apidata}{returns}
  \begin{datalist}
    \replyditem{A \texttt{GpsEquipment}, \texttt{CleaningEquipment} or
    \texttt{TransportEquipment} entity describing the equipment.}
    \replyitem{404}{If the team or the equipment is not found.}
  \end{datalist}
\end{apidata}
\begin{apidata}{access}
The logged in user must be an ``ADMIN'', the team's manager or a team member.
\end{apidata}


\subsection{Removing team equipment by index}

Given a team equipment index, this call deletes the equipment.
Valid indexes are in the range $0 \ldots n$, where $n$ is obtained by the call
described in Section~\ref{sec:team:eq_cnt}.

\apilocation{delete}{ws/team/\param{teamId}/equipment/\param{equipmentId}}
\begin{apidata}{returns}
  \begin{datalist}
    \replyitem{404}{If the team or the equipment is not found.}
  \end{datalist}
\end{apidata}
\begin{apidata}{access}
The logged in user must be an ``ADMIN'', the team's manager or a team member.
\end{apidata}


\subsection{Removing team equipment by ID}

Given an equipment identifier, this call removes the equipment from a team's
ownership. The equipment identifiers are a property of the \texttt{equipment}
entity (i.e., \texttt{equipmentId}). The latter must be obtained through one of
the calls described in Section~\ref{sec:team:eq_by_type} or
Section~\ref{sec:team:eq_by_idx}.

\apilocation{delete}{ws/team/\param{teamId}/equipmentId/\param{equipmentId}}
\begin{apidata}{returns}
  \begin{datalist}
    \replyitem{200}{If the operation succeeds.}
  \end{datalist}
\end{apidata}
\begin{apidata}{access}
The logged in user must be an ``ADMIN'' or the team's manager.
\end{apidata}


\subsection{Withdrawing a user from the member list}

\apilocation{delete}{ws/team/\param{teamId}/volunteer/\param{volunteerId}}
\begin{apidata}{returns}
  \begin{datalist}
    \replyitem{200}{If the operation succeeds.}
  \end{datalist}
\end{apidata}
\begin{apidata}{access}
The logged in user must be an ``ADMIN'' or the team's manager.
\end{apidata}



\subsection{Withdrawing an organization from the member list}

\apilocation{delete}{ws/team/\param{teamId}/organization/\param{volunteerId}}
\begin{apidata}{returns}
  \begin{datalist}
    \replyitem{200}{If the operation succeeds.}
  \end{datalist}
\end{apidata}
\begin{apidata}{access}
The logged in user must be an ``ADMIN'' or the team's manager.
\end{apidata}


\subsection{Assigning a new area to chart}

\apilocation{post}{ws/team/\param{teamId}/chartArea}
\begin{apidata}{contents}
  A \texttt{chartArea} entity whose only set field is the chart area ID.
\end{apidata}
\begin{apidata}{returns}
  \begin{datalist}
    \replyitem{200}{If the operation succeeds.}
    \replyitem{406}{An a string if constraints are violated and the assign
    operations is not accepted.}
  \end{datalist}
\end{apidata}
\begin{apidata}{access}
The logged in user must be an ``ADMIN'' or the team's manager.
\end{apidata}


\subsection{Removing an area from the list of areas to chart}

\apilocation{delete}{ws/team/\param{teamId}/chartArea/\param{chartAreaId}}
\begin{apidata}{returns}
  \begin{datalist}
    \replyitem{200}{If the operation succeeds.}
  \end{datalist}
\end{apidata}
\begin{apidata}{access}
The logged in user must be an ``ADMIN'' or the team's manager.
\end{apidata}


\subsection{Team report}

\apilocation{get}{ws/team/report?county=\param{county}}
\begin{apidata}{params}
  Omitting a query parameter behaves like a wildcard. All parameters can be
  repeated and the query will behave like a disjunction.
\end{apidata}
\begin{apidata}{returns}
  A list of teams matching the query criteria. The list can be formatted using
  XML, JSON, CSV, XLS or XLSX. To this end, you need to setup the appropriate
  MIME types in the HTTP Accept headers, i.e., ``application/json'',
  ``application/xml'', ``text/csv'', ``application/vnd.ms-excel'', respectively
  ``application/vnd.openxmlformats-officedocument.spreadsheetml.sheet''.
\end{apidata}
\begin{apidata}{access}
The logged in user must be an ``ADMIN'', ``ORGANIZER'', ``ORGANIZER\_MULTI''. 
\end{apidata}

\subsection{Assigned charted area report}

\apilocation{get}{ws/team/reportChartedArea?county=\param{county}\&\\chartedArea=\param{chartedAreaName}\&managerId=\param{managerId}}
\begin{apidata}{params}
  Omitting a query parameter behaves like a wildcard. All parameters can be
  repeated and the query will behave like a disjunction.
\end{apidata}
\begin{apidata}{returns}
  A list of teams matching the query criteria. The list can be formatted using
  XML, JSON, CSV, XLS or XLSX. To this end, you need to setup the appropriate
  MIME types in the HTTP Accept headers, i.e., ``application/json'',
  ``application/xml'', ``text/csv'', ``application/vnd.ms-excel'', respectively
  ``application/vnd.openxmlformats-officedocument.spreadsheetml.sheet''.

  The \texttt{chartedArea} objects assigned to each team in the result can be
  further filtered to match the query criteria using the
  \texttt{ro.ldir.dto.helper.AssignedChartedAreaFilter}. 
\end{apidata}
\begin{apidata}{access}
The logged in user must be an ``ADMIN'', ``ORGANIZER'', ``ORGANIZER\_MULTI''. 
\end{apidata}



\section{Garbage tag management}

This section describes the calls that affect the state of garbage tags inserted
in the system. These are implemented in the
\svnweb{ro/ldir/ws/GarbageWebService}{ro.ldir.ws.GarbageWebService} web
resource.

All calls to this web service must be authenticated using HTTP headers. The
username in the HTTP headers must match a user's email whose security role is
``ADMIN''.


\subsection{Adding a new garbage tag}

This call inserts a new garbage in the system. The owner of the garbage will be
set to match the authenticated user.

The town, county and charting area of the new garbage tag will be set
automatically by the system using the tag's coordinates and the entities stored
in the database.

\apilocation{post}{ws/garbage}
\begin{apidata}{contents}
  A \texttt{garbage} entity describing the new garbage tag.
\end{apidata}
\begin{apidata}{returns}
  \begin{datalist}
    \replyitem{200}{if the operation succeeds.}
    \replyitem{400}{if the garbage coordinates are wrong, i.e., they do not
    belong to any county.}
    \replyitem{500}{if another error occurs.}
  \end{datalist}
\end{apidata}


\subsection{Getting garbage tag information}

\apilocation{get}{ws/garbage/\param{garbageId}}
\begin{apidata}{returns}
  \begin{datalist}
    \replyditem{A \texttt{garbage} entity describing the garbage tag.}
  \end{datalist}
\end{apidata}

\subsection{Updating a garbage tag}

\apilocation{post}{ws/garbage/\param{garbageId}}
\begin{apidata}{contents}
  A \texttt{garbage} entity describing the new garbage tag.
\end{apidata}
\begin{apidata}{returns}
  \begin{datalist}
    \replyitem{200}{if the operation succeeds.}
    \replyitem{400}{if the garbage coordinates are wrong, i.e., they do not
    belong to any county.}
    \replyitem{500}{if another error occurs.}
  \end{datalist}
\end{apidata}

\subsection{Deleting a garbage tag}

\apilocation{delete}{ws/garbage/\param{garbageId}}
\begin{apidata}{returns}
  \begin{datalist}
    \replyitem{200}{if the operation succeeds.}
    \replyitem{500}{if another error occurs.}
  \end{datalist}
\end{apidata}



% \subsection{Deleting a garbage tag}
% 
% This method is not implemented yet.
% 
% \apilocation{delete}{ws/garbage/\param{garbageId}}
% \begin{apidata}{returns}
%   \begin{datalist}
%     \replyitem{200}{If the operation succeeds.}
%   \end{datalist}
% \end{apidata}
% \begin{apidata}{access}
% The logged in user must be an ``ADMIN'' or the garbage tag's owner.
% \end{apidata}


\subsection{Setting the status of a garbage tag}

\apilocation{put}{ws/garbage/\param{garbageId}/status}
\begin{apidata}{contents}
  A string denoting the new status of the garbage. Can be either ``CLEANED'' or
  ``IDENTIFIED''.
\end{apidata}
\begin{apidata}{returns}
  \begin{datalist}
    \replyitem{200}{if the operation succeeds.}
    \replyitem{500}{if another error occurs.}
  \end{datalist}
\end{apidata}
\begin{apidata}{access}
The logged in user must be an ``ADMIN'' or the garbage tag's owner.
\end{apidata}


\subsection{Assigning a new image to a garbage tag}

\apilocation{post}{ws/garbage/\param{garbageId}/image}
\begin{apidata}{contents}
  A ``multipart/form-data'' request containing a field ``file'' representing a
  new image to be associated to this garbage tag.
\end{apidata}
\begin{apidata}{returns}
  \begin{datalist}
    \replyitem{200}{if the operation succeeds.}
    \replyitem{404}{if the garbage tag does not exist.}
    \replyitem{500}{if another error occurs.}
  \end{datalist}
\end{apidata}
\begin{apidata}{access}
The logged in user must be an ``ADMIN'' or the garbage tag's owner.
\end{apidata}


\subsection{Getting an image of the garbage tag}

The following  calls retrieves an image associated to a garbage. All garbage
tags have an array of images, and new images are pushed to the tail of the
array. The array length can be obtained through the \texttt{pictures} field of
the \texttt{garbage} entity.  To make any of the following calls, a valid image
index must be known. Valid indexes are from 0 to \texttt{pictureCount} - 1.

To get an image in its original size, call:
\apilocation{get}{ws/garbage/\param{garbageId}/image/\param{image\_index}}
\begin{apidata}{returns}
  \begin{datalist}
    \replyditem{An ``image/jpeg'' file containing the picture.}
    \replyitem{404}{If the garbage tag does not exist or the index is invalid.}
  \end{datalist}
\end{apidata}

To get a lower resolution file, call:
\apilocation{get}{ws/garbage/\param{garbageId}/image/\param{image\_index}/display}
\begin{apidata}{returns}
  \begin{datalist}
    \replyditem{An ``image/jpeg'' file containing the picture.}
    \replyitem{404}{If the garbage tag does not exist or the index is invalid.}
  \end{datalist}
\end{apidata}

To get a thumbnail, call:
\apilocation{get}{ws/garbage/\param{garbageId}/image/\param{image\_index}/thumb}
\begin{apidata}{returns}
  \begin{datalist}
    \replyditem{An ``image/jpeg'' file containing the picture.}
    \replyitem{404}{If the garbage tag does not exist or the index is invalid.}
  \end{datalist}
\end{apidata}


\subsection{Deleting an image of the garbage tag}

This call deletes an image associated to a garbage. All garbage tags have an
array of images, and new images are pushed to the tail of the array. The array
length can be obtained through the \texttt{pictureCount} field of the
\texttt{garbage} entity. 

To make this call, a valid image index must be known. Valid indexes are from 0
to \texttt{pictureCount} - 1.

\apilocation{delete}{ws/garbage/\param{garbageId}/image/\param{image\_index}}
\begin{apidata}{returns}
  \begin{datalist}
    \replyitem{200}{If the operation succeeds.}
    \replyitem{404}{If the garbage tag does not exist or the index is invalid.}
  \end{datalist}
\end{apidata}
\begin{apidata}{access}
The logged in user must be an ``ADMIN'' or the garbage tag's owner.
\end{apidata}


\subsection{Searching for garbages in a given town}

\apilocation{get}{ws/garbage/townSearch/?town=\param{town\_name}}
\begin{apidata}{returns}
  \begin{datalist}
    \replyditem{A list of \texttt{garbage} entities of the garbages in the given
    town.}
  \end{datalist}
\end{apidata}


\subsection{Searching for garbages in a given county}

\apilocation{get}{ws/garbage/countySearch/?county=\param{county\_name}}
\begin{apidata}{returns}
  \begin{datalist}
    \replyditem{A list of \texttt{garbage} entities of the garbages in the given
    county.}
  \end{datalist}
\end{apidata}


\subsection{Searching for garbages by status}

\apilocation{get}{ws/garbage/statusSearch/?status=\param{status}}
\begin{apidata}{returns}
  \begin{datalist}
    \replyditem{A list of \texttt{garbage} entities of the garbages having the
    given status.}
  \end{datalist}
\end{apidata}

\subsection{Searching for garbages within a bounding box}

\apilocation{get}{ws/garbage/bbox/?topLeftX=\param{topLeftX}\&\\topLeftY=\param{topLeftY}\&bottomRightX=\param{bottomRightX}\&bottomRightY=\param{bottomRightY}}
\begin{apidata}{returns}
  \begin{datalist}
    \replyditem{A list of \texttt{garbage} entities of the garbages within the
    the provided bounding box.}
  \end{datalist}
\end{apidata}

\subsection{Garbage report}

\apilocation{get}{ws/garbage/report?county=\param{county}\&chartedArea=\param{name}\&\\userId=\param{ID}\&insertDate=\param{date}}
\begin{apidata}{params}
  Omitting a query parameter behaves like a wildcard in the query. All
  parameters can be repeated to obtain the result of a disjunction.
\end{apidata}
\begin{apidata}{returns}
  A garbage list matching the query criteria. The list can be formatted using
  XML, JSON, CSV, XLS or XLSX. To this end, you need to setup the appropriate
  MIME types in the HTTP Accept headers, i.e., ``application/json'',
  ``application/xml'', ``text/csv'', ``application/vnd.ms-excel'', respectively
  ``application/vnd.openxmlformats-officedocument.spreadsheetml.sheet''.
\end{apidata}
\begin{apidata}{access}
The logged in user must be an ``ADMIN'', ``ORGANIZER'', ``ORGANIZER\_MULTI''. 
\end{apidata}



\section{The geographical web service}

The system provides a geographical web service for managing the charting areas,
towns and counties known by the system. These are however to be set apriori
deployment, not to be changed after deployment, a feature were only ``ADMIN''
users have access and therefore are not documented. This is implemented in the
\svnweb{ro/ldir/ws/GeoWebService}{ro.ldir.ws.GeoWebService} web resource.

Documentation will be provided on a per ``need-to-know'' basis.

\subsection{Setting the percentage completed of a charted area}

\apilocation{put}{ws/geo/chartedArea/\param{chartedAreaId}/\\percentageCompleted}
\begin{apidata}{content}
  A \texttt{user} entity containing update information.
\end{apidata}
\begin{apidata}{returns}
  \begin{datalist}
    \replyitem{200}{if the operation succeeds.}
    \replyitem{403}{if the access policy if violated.}
    \replyitem{404}{if the charted area does not exists.}
    \replyitem{500}{if another error occurs.}
  \end{datalist}
\end{apidata}
\begin{apidata}{access}
The logged in user must be an ``ADMIN'' or charting the area.
\end{apidata}



